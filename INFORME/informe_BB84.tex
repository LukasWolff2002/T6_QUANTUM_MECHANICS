
\documentclass[12pt]{article}
\usepackage[utf8]{inputenc}
\usepackage[T1]{fontenc}
\usepackage[spanish]{babel}
\usepackage{amsmath, amssymb}
\usepackage{geometry}
\usepackage{graphicx}
\usepackage{hyperref}
\geometry{margin=2.5cm}

\title{\textbf{Algoritmo BB84 de Distribución Cuántica de Claves}}
\author{Manuel Tagle, Patricio Palacios y Lukas Wolff}
\date{\today}

\begin{document}
\maketitle

\section{Bases de Codificación}
El protocolo utiliza dos bases ortogonales mutuamente incompatibles para representar los bits:
\begin{itemize}
    \item Base \textbf{Rectilínea (R)}: $\{|0\rangle, |1\rangle\}$, correspondiente a polarizaciones horizontal y vertical.
    \item Base \textbf{Diagonal (D)}: $\{|+\rangle, |-\rangle\}$, correspondiente a polarizaciones a $45^\circ$ y $135^\circ$.
\end{itemize}
Un mismo bit puede codificarse en diferentes bases, y si el receptor mide en la base incorrecta, el resultado será aleatorio.

\section{Parte 1: Sin intervención de Eve}
\subsection{Descripción del Proceso}
El protocolo sin presencia de un espía se desarrolla en los siguientes pasos:

\begin{enumerate}
    \item \textbf{Generación de secuencias por Alice:} genera $N$ bits aleatorios ($0$ o $1$) y $N$ bases aleatorias (R o D). Si es R y el bit es $0$, envía $|0\rangle$; si es R y el bit es $1$, envía $|1\rangle$; si es D y el bit es $0$, envía $|+\rangle$; si es D y el bit es $1$, envía $|-\rangle$.
    \item \textbf{Bob genera sus bases:} elige $N$ bases aleatorias para medir los qubits recibidos. Ahora bien
    \item \textbf{Bob mide:} compara sus bases con las de Alice. Si coinciden, conserva el bit; si no, lo descarta. Las combinaciones posibles son las siguientes:
    \begin{itemize}
        \item Alice R, Bob R: Bob obtiene el bit correcto.
        \item Alice R, Bob D: Bob obtiene un bit aleatorio (50\% de probabilidad de ser $0$ o $1$).
        \item Alice D, Bob R: Bob obtiene un bit aleatorio (50\% de probabilidad de ser $0$ o $1$).
        \item Alice D, Bob D: Bob obtiene el bit correcto.
        \item \textbf{Perdida de información:} en promedio, la mitad de los bits se descartan debido a la falta de coincidencia de bases, es decir, hay un $50\%$ de perdida de información, el cual tiene un $50\%$ de probabilidad de ser real, por lo tanto, se puede decir que hay un $25\%$ de perdida de información real. Aun asi, se descarta la mitad de los bits, logrando asi un Quantum Bit Error Rate (QBER) de $0\%$ en un canal ideal.
    \end{itemize}
    \item \textbf{Comparación de bases:} finalmente, Alice publica sus bases y Bob determina las coincidencias.
    \item \textbf{Extracción de clave cruda:} se seleccionan los bits correspondientes a bases coincidentes.
    \item \textbf{Verificación de errores:} Bob revela una pequeña fracción de la clave para comparar con Alice. Si los bits coinciden, la clave es válida.
\end{enumerate}

\subsection{Resultados Esperados}
Si el canal es ideal (sin ruido), la tasa de error cuántica o QBER (\textit{Quantum Bit Error Rate}) debería ser aproximadamente $0\%$.
La mitad de los bits se descartan debido a la falta de coincidencia de bases, resultando en una \textbf{clave cruda} de longitud cercana a $N/2$.

\section{Parte 2: Con intervención de Eve}
\subsection{Ataque de Interceptación y Reenvío}
En este escenario, un atacante (Eve) intercepta los qubits enviados por Alice, los mide en bases aleatorias y luego reenvía a Bob qubits codificados según sus propios resultados. 
El procedimiento es el siguiente:

\begin{enumerate}
    \item Eve genera sus bases aleatorias.
    \item Eve mide los qubits de Alice. Si su base coincide con la de Alice, obtiene el bit correcto; si no, el resultado es aleatorio.
    \item Eve reenvía a Bob los bits medidos, preparados en su base.
    \item Bob mide los qubits reenviados utilizando sus propias bases. Por lo cual se generan las siguientes combinaciones:
    \begin{itemize}
        \item Alice R, Eve R, Bob R: Bob obtiene el bit correcto.
        \item Alice R, Eve R, Bob D: Bob obtiene un bit aleatorio (50\% de acierto).
        \item Alice R, Eve D, Bob R: Bob obtiene un bit aleatorio (50\% de acierto).
        \item Alice R, Eve D, Bob D: Bob obtiene el bit correcto con 50\% de probabilidad.
        \item Alice D, Eve R, Bob R: Bob obtiene el bit correcto con 50\% de probabilidad. % <-- corregido
        \item Alice D, Eve R, Bob D: Bob obtiene un bit aleatorio (50\% de acierto).
        \item Alice D, Eve D, Bob R: Bob obtiene el bit correcto con 50\% de probabilidad.
        \item Alice D, Eve D, Bob D: Bob obtiene el bit correcto.
    \end{itemize}
\end{enumerate}

\subsection{Detección de la Presencia de Eve}
El intento de espionaje introduce perturbaciones detectables. Si Eve mide con una base incorrecta, perturba el estado original, 
de modo que cuando Bob mida (incluso con la base correcta), tiene un 50\% de probabilidad de obtener el bit erróneo.
El resultado es que la tasa de error esperada (\textbf{QBER}) se eleva aproximadamente a un \textbf{25\%}.

\end{document}

